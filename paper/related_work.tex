\section{Related Work} \label{relatedwork}

The attribute grammar, originally developed in 1968 by Knuth \cite{knuth1968semantics} in context of compiler
construction, has been successfully used as a tool for design, formal specification
and implementation of practical systems. It has influenced areas of
Computer Science such as natural language processing (\citealp{hafiz2011modular, starkie2002inferring}), 
definite clause grammars \cite{bratko2001prolog}, query processing (\citealp{koch2007attribute,ramakrishnan1991top}) and specification of algorithms \cite{bellanova1984examples}.
Thirunarayan \cite{thirunarayan2009attribute} provides a good overview of 
applications, including static analysis of programs, program translation, specifying information
extraction algorithms and optimization of datalog programs.

In our work, we apply attribute grammars to an optimization problem. This has previously been explored in range of domains: from well-known algorithmic problems 
like knapsack packing \cite{o2004solving}, through bioinformatics \cite{waldispuhl2002approximate} to music domain\cite{desainte1994using}.
However, we are not aware of any previous work related to discovering mathematical formulas. The closest work to ours can be found in \cite{cheung1999attribute} which involves searching
over the space of algorithms and the grammar attributes are also computational complexity.



