Attribute grammar, originally developed in 1968 by Knuth \cite{knuth1968semantics} in context of compiler
construction, was successfully used as a tool for design, formal specification
and implementation of practical systems. It influenced various areas of
Computer Science such as natural language processing \cite{hafiz2011modular}, \cite{starkie2002inferring}, 
definite clause grammars \cite{bratko2001prolog}, query processing \cite{koch2007attribute}, \cite{ramakrishnan1991top} and specification of algorithms \cite{bellanova1984examples}.
Thirunarayan \cite{thirunarayan2009attribute} provides a good overview of their
applications, including static analysis of programs, program translation, specifying information
extraction algorithms and optimization of datalog programs.

In our work, we focus on applying attribute grammar for optimization problem. This class
of problems was successfully addressed in range of domains: from well-known algorithmic problems 
like knapsack packing \cite{o2004solving}, through bioinformatics \cite{waldispuhl2002approximate} to music domain\cite{desainte1994using}.
We are not aware of any previous work related to discovering mathematical formulas. However,
similar concepts can be found in \cite{cheung1999attribute} where authors are searching
the space of algorithms and attributes are also computational complexity.



