\section{Discussion and Future Work}

We have presented an automatic method for discovery of computations for
 polynomial expressions. The method itself is novel, as well as the derived
approximations to the RBM partition function. There are four
major directions of future research. 


The first is focus on extension of our method
to (i) richer class of functions than polynomials, (ii) larger set of production
rules, (iii) more complex objects than matrices (e.g. tensors).


The second would be to use learning to
generalize computation (Subsection \label{agenda)}, i.e.~replace current brute force process with
something more intelligent. E.g. We could generalize computation of
$g(x \rightarrow x^k, W)$ for $k = 1, \dots, 6$ to $k > 6$. 


The third is concerned with computing
the partition function and its derivatives, using approximations for
learning, approximations for deep Boltzmann machines, optimization of
the derived formulas, and understanding the mathematical principles behind
formulas derived by our programs. 


Finally,
the forth direction are real-world applications of this framework on
large code databases. We could automatically detect parts of code with
expressions that have suboptimal time complexity and replace them.


We discuss in following subsections two first major directions of future research.


\subsection{Extension of Grammar}
The presented framework has some limitations, but also can be easily extended. First,
we operate on polynomials instead of general functions. It is quite easy to verify
equality of polynomials, because it is enough to check equality of coefficients or
evaluations on sufficiently many points (more than degree of
polynomial). However, for generic functions
it is not as simple. For example, if our basis would to include trigonometric functions, then the
framework would have to be aware of many additional identities,
e.g. $\sin^2{x} + \cos^2{x} = 1$, or $2\sin x \cos x = \sin
2x$. Although challenging, we believe it is possible to extend our
framework to handle such cases. 

It is straight forward to extend our framework to matrices of fractional
polynomials instead of matrices of polynomials.  We could
then include productions like matrix inverse and element-wise inverse of
a matrix.  Another direction would be to generalize matrices to arbitrary tensors
for which it is even harder for humans to spot useful relations.

Finally, we are interested in exploiting productions which could discover 
recurrences. This would allow the discovery the fast Fourier transform, or
Strassen's algorithm for fast matrix multiplication. This family of problems is quite broad,
and crucial in computation.


\subsection{Computation Generalization}\label{agenda}

This paper presents a brute force search over all possible computations, 
to yield terms that could lead to the target result. Although it works
well for polynomials of small degrees $k \leq 6$, for larger powers 
brute force methods seems to fail. One of the contributions of this paper is to
demonstrate how a small subset of mathematics may be explored with an automatic proving
system. However, to broaden the range of math that we can address, we
must move beyond brute force methods to more intelligent approaches
which learn which rules are likely to be helpful, based on expressions
for smaller powers. This would allow us move to polynomials of $k > 6$,
making accurate estimation of the partition function for many deep
networks possible. More broadly, it would bring machine learning techniques
to the area of automatic theorem proving, which is another domain of
human intelligence distinct from perceptual tasks (where machine
learning is already effective). 

%  tackle a broader set of problems We look forward to extend this brute force methods to more intelligible approaches, which
% could learn based on rules for smaller powers, how to solve problem for larger powers. This would
% have two-fold implications in machine learning. Firstly, it would bring machine learning techniques
% to the area of automatic theorem proving, which another domain of human intelligence after perception
% of vision, or hearing. Secondly, solution to the problem of partition function approximation
% could have consequences in unsupervised learning of RBMs, deep Bolzmann machines, and deep
% belief networks. 



