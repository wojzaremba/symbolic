\section{Terminology}
We define the following vocabulary: 

\begin{description}
  \item[Space of matrices of polynomials\label{itm:spacematrices}] \hfill \\ A matrix of homogeneous polynomials, which we denote by $\mathcal{P}^{n \times m}_{\alpha}$. Upper index indicates size of matrix, and lower indicates degree. For instance, $\begin{pmatrix} a^3 + b^3 & b^3 + bc^2 + c^3\\ cd^2 & d^3 \end{pmatrix} \in \mathcal{P}^{2 \times 2}_3$. 

\item[Grammar\label{itm:grammar}] \hfill \\ An attributive grammar, with \ref{itm:production} defined on \ref{itm:expression}s and semantic rules on \ref{itm:attribute}s. The attributes associated to every \ref{itm:expression}, are: \ref{itm:computation}, \ref{itm:time} and \ref{itm:term}.

\item[Production rules\label{itm:production}] \hfill \\ Syntactic rules defined on \ref{itm:expression}s (transformation \ref{itm:expression} $\rightarrow$ \ref{itm:expression}), with associated semantic rules on \ref{itm:attribute}s. They are defined in figures \ref{fig:rules2} and \ref{fig:rules}.

\item[Expression\label{itm:expression}] \hfill \\ Matrices of polynomials having a homogeneous degree $\alpha$. They belong to the spaces $\mathcal{P}^{n \times m}_\alpha$. Different expressions might be in different spaces, and it restricts which production rules can be applied. Every expression has associated \ref{itm:attribute}s

\item[Attribute\label{itm:attribute}] \hfill \\
  A value associated with \ref{itm:expression}. 

\item[Computation\label{itm:computation}] \hfill \\ A piece of code to compute given \ref{itm:production} on particular architecture or programming language. In our rules we consider Matlab as
  underlying computation language. For example, in Matlab the transpose
  operation on matrix $\mathbb{A}$ is denoted as $\mathbb{A}'$, so our computation
  for transpose production is $\mathbb{A}'$. If we ever decide to implement our framework in R,
  the Computation for transpose would be $t(\mathbb{A})$. Computation is an \ref{itm:attribute}.

\item[Time\label{itm:time}] \hfill \\ A computation time for a particular architecture and programming language (seconds), or computation complexity (polynomial). It is an \ref{itm:attribute}.

\item[Term\label{itm:term}] \hfill \\ An instance of \ref{itm:expression}. It is an \ref{itm:attribute}.

%Let's consider a literal $L = (\mathbb{E}, \mathcal{C}, t_\mathcal{C})$. $\mathbb{E}$ is a polynomial in $\mathcal{P}^{n \times m}_{\alpha}$. $\mathcal{C}$ is a computation of this polynomial starting with a input matrices. Result of computation is a real matrix
%in $\mathbb{R}^{n \times m}_k$, which is computed in time $t_\mathcal{C}$. This time is a time necessary to perform computation on instantiations of matrices with actual values -- not computation during grammar derivation, which would involve expressions. For example, if $\mathcal{C}$ is an elementwise multiplication operation (see rules in \ref{rules}) then $t_\mathcal{C}$ would be $O(nm)$.

\end{description}
