Presented in this paper method is a brute force over all possible computations, 
which could lead to the target result. It seems to work reasonably well
for polynomials of small degrees $k \leq 5$, however for larger powers 
brute force methods seems to fail. One of major goals of this paper is to
bring a small subset of mathematics, which has tractable brute force automatic proving
system. We look forward to extend this brute force methods to more intelligible approaches, which
could learn based on rules for smaller powers, how to solve problem for larger powers. This would
have two-fold implications in machine learning. Firstly, it would bring machine learning techniques
to the area of automatic theorem proving, which another domain of human intelligence after perception
of vision, or hearing. Secondly, solution to the problem of partition function approximation
could have consequences in unsupervised learning of RBMs, deep Bolzmann machines, and deep
belief networks. 

