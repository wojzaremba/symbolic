This paper presents a brute force search over all possible computations, 
to yield terms that could lead to the target result. Although it works
well for polynomials of small degrees $k \leq 5$, for larger powers 
brute force methods seems to fail. One of the contributions of this paper is to
demonstrate how a small subset of mathematics may be explored with an automatic proving
system. However, to broaden the range of math that we can address, we
must move beyond brute force methods to more intelligent approaches
which learn which rules are likely to be helpful, based on expressions
for smaller powers. This would allow us move to polynomials of $k>5$,
making accurate estimation of the partition function for many deep
networks possible. More broadly, it would bring machine learning techniques
to the area of automatic theorem proving, which is another domain of
human intelligence distinct from perceptual tasks (where machine
learning is already effective). 

%  tackle a broader set of problems We look forward to extend this brute force methods to more intelligible approaches, which
% could learn based on rules for smaller powers, how to solve problem for larger powers. This would
% have two-fold implications in machine learning. Firstly, it would bring machine learning techniques
% to the area of automatic theorem proving, which another domain of human intelligence after perception
% of vision, or hearing. Secondly, solution to the problem of partition function approximation
% could have consequences in unsupervised learning of RBMs, deep Bolzmann machines, and deep
% belief networks. 

