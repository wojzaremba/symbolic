\begin{table}[t]
\tiny
\centering
\begin{tabular}{rrr}
\hline
Degree & Num. terms & Complexity \\
\hline
2 & 4 & $O(n^2)$\\
3 & 5 & $O(n^2)$\\
4 & 21 & $O(n^3)$\\
5 & 30 & $O(n^3)$\\
6 & 106 & $O(n^3)$\\
\hline
\end{tabular}
\caption{A summary of the complexity of computation for $g(x \rightarrow x^k, W)$.
% KAROL-CUT
%  The number of terms, and the largest complexity among them, is important for the computation of the target expression.
  Potentially, we need to evaluate partition function multiple times, so the ``Complexity''
  is the complexity of our final system which exctly reproduces the
  Taylor-series approximation, the naive computation of which would be
  exponential $n$ (see Figure \ref{time_approx}).} 
\label{eval}
\vspace{-4mm}
\end{table}

