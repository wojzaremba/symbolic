\subsection{Dropout marginalization}\label{sec:dropout}
In this section we will perform analogous analysis to the one from Subsection \ref{subsubsec:gx} 
for a single layer network with $L_2$ loss. 
First, we derive update rules for weights for a single layer neural networks without any regularization. Then
we will confront it with dropout network. We will compute marginalization over
all possible dropout masks.
We will describe how our framework can discover
such update rules for more complex networks.

Single layer neural network with $L_2$ loss minimizes following objective:
\begin{equation*}
  \min_W \frac{1}{2} ||WX - Y||^2, X \in \mathbb{R}^{f \times b}, Y \in \mathbb{R}^{g \times b}, W \in \mathbb{R}^{g \times f}
\end{equation*}
By differentiating above equation one gets gradient descent update rule
on $W$:
\begin{equation*}
 \partial W = (WX - Y)X^T  = WXX^T - YX^T
\end{equation*}
We note for further derivation that $C = XX^T$ is the covariance matrix of
$X$, and that $C_{i,j} = \sum_k X_{i, k}X_{j, k}$.
Dropout minimizes following objective:
\begin{equation*}
  \argmin_W \mathbb{E}_M \frac{1}{2}||W(X .* M) - Y||^2, M \in \{0, 1\}^{f \times b}
\end{equation*}
where $.*$ is element-wise multiplication and the expectation operator
acts over all possible mask assignments. For simplicity, we assume
that we drop neurons with probability $0.5$. Then expectation can then
be replaced with a sum over all possible masks. We then compute the
derivative with respect to weights $W$.
\begin{align*}
  \partial W & = \sum_M (W(X .* M) - Y)(X .* M)^T \\
  & = \sum_M W(X .* M)(X .* M)^T - Y(X .* M)^T\\
  & = W\sum_M [(X .* M)(X .* M)^T] - 2^{fb - 1}YX^T\\
\end{align*}
Let  $D = \sum_M (X .* M)(X .* M)^T$. 
\begin{align*}
  D_{i,j} & = \sum_M \sum_k X_{i, k} X_{j, k} M_{i, k} M_{j, k} \\
  D_{i,j, i \neq j} & = 2^{fg - 2} \sum_k X_{i, k} X_{j, k} = 2^{fg - 2}C_{i, j} \\
  D_{i, i} & = 2^{fg - 1} \sum_k X_{i, k} X_{i, k} = 2^{fg - 1}C_{i, i} \\
  D & = 2^{fg - 2}(XX^T + XX^T .* I) \\
  \partial W & = \frac{1}{2}W (XX^T + XX^T .* I) - YX^T
\end{align*}
where $I$ is the identity matrix. 


% The above derivation of dropout
% marginalization can be discovered by our framework, and can be
% generalized to multiple layer network with polynomial activation
% functions. 

The derivation above has been presented in several previously works
analyzing dropout \cite{chen2013learning, chen2012marginalized, maaten2013learning}. 
However, these just consider a single layer network with variations in loss function, and noise type, 
or approximate schemas. \cite{baldi2014dropout} considers exact marginalization 
for deep linear networks, but approximate marginalization for non-linear activation functions.

Our framework is not only capable of discovering the derivation above,
but can also handle the multi-layer case, as typically used in big networks:

% In our framework, we can jointly compute update rule to weights for
% multiple layers.


 
% Potentially, dropout from one layer can influence results on other layers. It is not enough
% to consider each layer in separation to derive proper dropout marginalization.


% Let's consider two layer neural network:
\begin{equation*}
  \argmin_{W_1, W_2} \mathbb{E}_{M_1, M_2} \frac{1}{2}||W_2(f(W_1(X .* M_1)) .* M_2) - Y||^2
\end{equation*}
In our analysis, we consider the function $f$ to be just linear 
function.  Applying our approach, we find that the closed-form derivatives for $W_1$ and $W_2$ for marginalized 
dropout are:
\begin{lstlisting}
I1 = eye(F);
I2 = eye(G);
C = X * X';
D = W2' * W2;
dW1 = D * W1 * C + ...
(D .* I2) * W1 * C + ...
(D .* I2) * W1 * (C .* I1) + ...
D * W1 * (C .* I1) - ...
4 *(W2' * Y) * X'; 
dW2 = W2 * W1 * C * W1' + ...
W2 * W1 * (C .* I1) * W1' + ...
W2 * ((W1 * (C .* I1) * W1') .* I2) + ...
W2 * ((W1 * C * W1') .* I2) - ...
4 * Y * X' * W1';
\end{lstlisting}

In future work, we intend to extend our analysis of dropout to
non-linear functions $f$. The closed-form rules might
allow us to understand why dropout works, and how to improve it. We
would also like to  employ our framework for more challenging neural network architectures involving
convolutions.
