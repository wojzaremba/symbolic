\subsection{Dropout marginalization}\label{sec:dropout}
In this section we will perform analogous analisys to the one from Subsection \ref{subsubsec:gx} 
for a single layer network with $L_2$ loss. 
First, we derive update rules for weights for a single layer neural networks without any regularization. Then
we will confront it with dropout network. We will compute marginalization over
all possible dropout masks.
We will describe how our framework can discover
such update rules for more complex networks.

Single layer neural network with $L_2$ loss minimizes following objective:
\begin{equation*}
  \min_W ||WX - Y||^2, X \in \mathbb{R}^{f \times b}, Y \in \mathbb{R}^{g \times b}, W \in \mathbb{R}^{g \times f}
\end{equation*}
By differentiating above equation one gets gradient descent update rule
on $W$ (sign $\sim$ indicates that we drop multiplicative constants)
\begin{equation*}
 \partial W = 2(WX - Y)X^T \sim WXX^T - YX^T
\end{equation*}
We note for further derivation that $C = XX^T$ is the covariance matrix of
$X$, and that $C_{i,j} = \sum_k X_{i, k}X_{j, k}$.


Dropout minimizes following objective:
\begin{equation*}
  \argmin_W \mathbb{E}_M ||W(X .* M) - Y||^2, M \in \{0, 1\}^{f \times b}
\end{equation*}
Above expectation operator acts over all possible mask assignments, and $.*$ is 
element-wise multiplication. For simplification
let's assume that we drop neurons with probability $0.5$. Expectation
can be replaced with sum over all possible masks. Sd compute
derivative with respect to weights $W$. 
\begin{align*}
  \partial W \sim \sum_M (W(X .* M) - Y)(X .* M)^T = \\
  \sum_M W(X .* M)(X .* M)^T - Y(X .* M)^T = \\
  W\sum_M [(X .* M)(X .* M)^T] - 2^{fb - 1}YX^T= \\
\end{align*}
Let's denote $D = \sum_M (X .* M)(X .* M)^T$. 
\begin{align*}
  D_{i,j} = \sum_M \sum_k X_{i, k} X_{j, k} M_{i, k} M_{j, k} \\
  D_{i,j} \text{ for } i \neq j = 2^{fg - 2} \sum_k X_{i, k} X_{j, k} = 2^{fg - 2}C_{i, j} \\
  D_{i,j} \text{ for } i = j = 2^{fg - 1} \sum_k X_{i, k} X_{j, k} = 2^{fg - 1}C_{i, j} \\
  D = 2^{fg - 2}(XX^T + XX^T .* I) \text { $I$ is identity matrix } \\
  \partial W \sim \frac{1}{2}W (XX^T + XX^T .* I) - YX^T
\end{align*}

Above derivation of dropout marginalization can be discovered by our framework, and can be generalized
to multiple layer network with rational (quotient of two polynomials) activation functions. Our framework
has to be start with following start symbols: $W, X, Y, I$ (one needs additional symbol for identity).


Our framework is able to discover update weights $W$ for networks with more than one layer. 
For instance, it is able to find update rule for two-layer neural network objective:
\begin{equation*}
  \argmin_{W_1, W_2} \mathbb{E}_{M_1, M_2} ||W_2(f(W_1(X .* M_1)) .* M_2) - Y||^2
\end{equation*}
$f$ in this equation is a rational function. 


% \section{Data augmentation marginalization}\label{sec:augm}
% We discuss in this section how to marginalize augmentation by change of contrast. 
% Contrast is changed according to normal distribution with variance, which was precomputed.


% Let $A$ be a random variable describing how we want to alter contrast. Linear regression
% with contrast alternation can be described as: 
% \begin{align*}
% & \argmin_{W} \mathbb{E}_A ||W(X + A) - Y||^2\\
% \end{align*}
% Usually contrast alternation is described with continuous distribution. It means that
% expected value will be replaced with integral. Let's assume for simplicity that mean 
% of contrast alternation is zero.
% \begin{align*}
% \partial W & \sim \int_A (W(X + A) - Y)(X + A)^T \\
%    & = \int_A (W(XX^T + AA^T) - YX^T \\
%    & = W(XX^T + \int AA^T) - YX^T 
% \end{align*}
% Update is modified by covariance matrix of contrast matrix. To discover computation,
% which involves contrast alternation our framework would have to have access to 
% expressions involving moments of $A$, and to starting symbols like $W, X, Y$.
% Above derivation could be generalized to multilayer neural network. This would
% give a single update rule consisting with respect to all possible alternated
% input images.

