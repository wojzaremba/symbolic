Algorithm presented in Section
\ref{sec:grammars} allows to find concise formulas for polynomial expressions.
However, many interesting functions are outside of this family.  Instead of it,
we are going to consider Taylor expansion of desired function, and we will
derive close form fast formula for it.

Let's denote by $g(f, W)$ generalization of partition function. 
$g$ is a functional, and it takes as argument a function $f : \mathbb{R} \rightarrow \mathbb{R}$,
and weights $W$. It is defined as follow: 
\begin{gather*}
g(f, W) = \sum_{v \in \{0, 1\}^n, h \in \{0, 1\}^m} f(v^TWh) \\
f : \mathbb{R} \rightarrow \mathbb{R}\\
W \in \mathbb{R}^{n \times m}
\end{gather*}
We consider computation of $g(x \rightarrow x^k, W)$ for a given $k$ for any $W
\in \mathbb{R}^{n \times m}$, for any $n, m$. Potentially, if we would be able
to compute $g(x \rightarrow x^k, W)$ for $k = 1, \dots, K$, than partition
function for finite energy $v^TWh < C$ could be approximated arbitrarily well.
This is consequence of expressing $e^{x}$ as finite sum approximation through
Taylor expansion.

\subsection{Low degree examples} In order to present how our algorithm works,
we will manually derive fast computation procedure for $g(x \rightarrow k, W)$.
However, this can be done manually only for very small $k = 1, 2$. 


\subsubsection{$g(x \rightarrow x, W)$} Let's consider function $f(x) = x$. We
will show that function $g(x \mapsto x, W)$ is computable in $O(nm)$ time
(linear with respect to number of entries in $W$ matrix).
\begin{gather*}
	g(x \mapsto x, W) = \sum_{v \in \{0, 1\}^n, h \in \{0, 1\}^m} v^TWh
\end{gather*}
Entry $w_{i,j}$ in above sum is counted only if $v_i = 1$ and $h_j = 1$. Other variables
$v_1, \dots v_{i-1}, v_{i+1}, \dots v_n$ and $h_1, \dots h_{i-1}, h_{j+1}, \dots h_m$ can be 
assigned arbitrary. Number of arbitrary assignments of remaining variables is $2^{n + m - 2}$. 
This concludes that:
\begin{gather*}
	\sum_{v \in \{0, 1\}^n, h \in \{0, 1\}^m} v^TWh = 2^{n + m - 2}\sum_{i = 1, \dots, n, j = 1, \dots, m} W_{i, j}
\end{gather*}
Above mathematical formula (or description of computation) is the close form solution for sum over exponentially many elements.
Complexity of it is just linear in size of $W$, which is $O(n^2)$.

\subsubsection{$g(x \rightarrow x^2, W)$}

We are interest in computing following expression: 
\begin{gather*}
	g(x \mapsto x^2, W) = \sum_{v \in \{0, 1\}^n, h \in \{0, 1\}^m} (v^TWh)^2
\end{gather*}

There are multiple second order monomials that emerge: 

\begin{itemize}
	\item $w_{i,j}^2$ - present iff $v_i = 1, h_j = 1$. Appears $2^{n + m - 2}$ times.
	\item $w_{i,j} w_{i, k}, j \neq k$ - present iff $v_i = 1, h_j = 1, h_k = 1$. Appears $2^{n + m - 3}$ times.	
	\item $w_{i,j} w_{k, j}, i \neq k$ - present iff $v_i = 1, v_k = 1, h_j = 1$. Appears $2^{n + m - 3}$ times.
	\item $w_{i,j} w_{k, l}, i \neq k, j \neq l$ - present iff $v_i = 1, v_k = 1, h_j = 1, h_l = 1$. Appears $2^{n + m - 4}$ times.			
\end{itemize}
We encode above quantities in a vector, which indicate how many times particular monomials 
appear. Vector expressing this relation for $g(x \mapsto x^2, W)$ is $(2^{n + m - 2}, 2^{n + m - 3}, 2^{n + m - 3}, 2^{n + m - 4})$


Let us consider following expressions: 
\begin{itemize}
 \item $\sum_{i = 1, \dots, n, j = 1, \dots m} W_{i, j}^2$ encodes $(1, 0, 0, 0)$. 
 \item $(\sum_{i = 1, \dots, n, j = 1, \dots m} W_{i, j})^2$ encodes $(1, 1, 1, 1)$.
 \item $\sum_{i = 1, \dots, n}(\sum_{j = 1, \dots, m} W)^2$ encodes $(1, 1, 0, 0)$. 
 \item $\sum_{j = 1, \dots, m}(\sum_{i = 1, \dots, n} W)^2$ encodes $(1, 1, 0, 0)$. 
\end{itemize}
 
 By solving linear equation:
 \begin{equation}
 \begin{pmatrix} 
  1 & 1 & 1 & 1 \\ 
  0 & 1 & 1 & 0 \\ 
  0 & 1 & 0 & 1 \\ 
  0 & 1 & 0 & 0 \\     
\end{pmatrix}t = 2^{n + m - 4}\begin{pmatrix} 
  2^{2}\\ 
  2^{1}\\ 
  2^{1}\\ 
  2^{0}\\     
\end{pmatrix} \\
 \end{equation}

One can find that 
\begin{align*}
	&g(x \mapsto x^2, W) = 2^{n + m - 4} \\ 
 &\Big(\sum_{i = 1, \dots, n, j = 1, \dots m} W_{i, j}^2 + (\sum_{i = 1, \dots, n, j = 1, \dots m} W_{i, j})^2 + \\
 &\sum_{i = 1, \dots, n}(\sum_{j = 1, \dots, m} W)^2 + \sum_{j = 1, \dots, m}(\sum_{i = 1, \dots, n} W)^2 \Big)
\end{align*}
Above derivation is still in scope of human skills. However, manual derivation
of $g(x \mapsto x^k, W)$ for further $k > 2$  seems to be a feet. Our algorithm
is able to find such complex computational patterns automatically. 

\subsubsection{$g(x \rightarrow x^3, W)$}
Manual derivation for $k = 3$ seems to be a hard task. We present here all derived computations in
for expressions in $\mathcal{P}^{1 \times 1}_3$. 

\begingroup
    \fontsize{8pt}{12pt}\selectfont
\begin{itemize}
	\itemsep-5px 
	\item \begin{verbatim} sum(sum(W, 2), 1) \end{verbatim}
	\item \begin{verbatim} (sum(sum(W, 2), 1) .* sum(sum(W, 2), 1)) \end{verbatim}
	\item \begin{verbatim} sum((sum(W, 1) .* sum(W, 1)), 2) \end{verbatim}
	\item \begin{verbatim} sum(sum(((repmat(sum(W, 2), [1, m]) .*  ...
		repmat(sum(W, 1), [n, 1])) .* W), 2), 1) \end{verbatim}
	\item \begin{verbatim} sum((sum(W, 1) .* (sum(W, 1) .* sum(W, 1))), 2) \end{verbatim}
	\item \begin{verbatim} sum((sum(W, 2) .* sum(W, 2)), 1) \end{verbatim}
	\item \begin{verbatim} sum((sum(W, 2) .* (sum(W, 2) .* sum(W, 2))), 1) \end{verbatim}
	\item \begin{verbatim} sum(sum((W .* W), 2), 1) \end{verbatim}
	\item \begin{verbatim} sum(sum(W, 2), 1) .^ 3 \end{verbatim}
	\item \begin{verbatim} sum(sum((repmat(sum(W, 1), [n, 1]) .^ 2  .* ...
		 (repmat(sum(W, 2), [1, m]))), 1), 2) \end{verbatim}
	\item \begin{verbatim} sum((sum(W, 1) .* sum((W .* W), 1)), 2) \end{verbatim}
	\item \begin{verbatim} (sum(sum(W, 2), 1) .* sum(sum((W .* W), 2), 1)) \end{verbatim}
	\item \begin{verbatim} sum(sum((repmat(sum(W, 2), [1, m]) .^ 2 .* ...
		repmat(sum(W, 1), [n, 1]))), 2), 1) \end{verbatim}
	\item \begin{verbatim} sum((sum(W, 2) .* sum((W .* W), 2)), 1) \end{verbatim}
	\item \begin{verbatim} sum(sum((W .* (W .* W)), 2), 1) \end{verbatim}
\end{itemize}
\endgroup


Expression $g(x \rightarrow x^3, W)$ can be computed exactly with:

\begingroup
    \fontsize{8pt}{12pt}\selectfont
\begin{verbatim}
((((sum(sum(W, 2), 1) .* (sum(sum(W, 2), 1) ...
.* sum(sum(W, 2), 1)))) + ((sum(sum(W, 2), 1) ...
.* sum((sum(W, 2) .* sum(W, 2)), 1)) .* 3) ...
+ ((sum(sum(W, 2), 1) ...
.* sum(sum((W .* W), 2), 1)) .* 3) ...
+ ((sum(sum(W, 2), 1) .* sum((sum(W, 1) ...
.* sum(W, 1)), 2)) .* 3) + (sum((sum(W, 2) ...
.* sum((W .* repmat(sum(W, 1), ...
[n, 1])), 2)), 1) .* 6))) / 64;
\end{verbatim}
\endgroup



