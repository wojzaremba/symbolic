\section{Partition Function of RBM} \label{partitionfunction}

The algorithm presented in Section
\ref{sec:grammars} allows us to find concise formulas for polynomial expressions.
However, many interesting functions are outside of this
family. Therefore we consider Taylor expansion of desired function
and use our approach to derive a fast way of computing the expansion
in closed form.

Let $g(f, W)$ be the generalization of partition function for a binary
RBM \cite{hinton2002training}. Let $f : \mathbb{R} \rightarrow \mathbb{R}$ and $W \in \mathbb{R}^{n \times m}$,
we define a functional $g$ as follows: \\
\begin{gather*}
g(f, W) = \sum_{v \in \{0, 1\}^n, h \in \{0, 1\}^m} f(v^TWh)
\end{gather*}

We consider the computation of $g(x \rightarrow x^k, W)$ for a given power
$k$, and for any $W
\in \mathbb{R}^{n \times m}$ (and any size $n, m$). Potentially, if we would be able
to compute $g(x \rightarrow x^k, W)$ for $k = 1, \dots, K$, than partition
function for finite energy $v^TWh < C$ could be approximated arbitrarily well.
This is consequence of expressing as a finite sum approximation through
Taylor expansion: $e^{x}=1+x+x^2/2!+x^3/3!+\cdots$.

\subsection{Low degree examples} In order to present how our algorithm works,
we will manually derive a fast computation procedure for $g(x \rightarrow x^k, W)$.
However, this can be done manually only for very small $k = 1, 2$. 


\subsubsection{{$\bf g(x \rightarrow x, W)$}} Let's consider function $f(x) = x$. We
will show that function $g(x \mapsto x, W)$ is computable in $O(nm)$ time
(i.e.~linear with respect to number of entries in $W$ matrix).
\begin{gather*}
	g(x \rightarrow x, W) = \sum_{v \in \{0, 1\}^n, h \in \{0, 1\}^m} v^TWh
\end{gather*}
An entry $w_{i,j}$ in the sum is counted only if $v_i = 1$ and $h_j = 1$. Other variables
$v_1, \dots v_{i-1}, v_{i+1}, \dots v_n$ and $h_1, \dots h_{i-1}, h_{j+1}, \dots h_m$ can be 
assigned arbitrarily, with the number of arbitrary assignments being
$2^{n + m - 2}$. Hence:
\begin{gather*}
	\sum_{v \in \{0, 1\}^n, h \in \{0, 1\}^m} v^TWh = 2^{n + m - 2}\sum_{i = 1, \dots, n, j = 1, \dots, m} W_{i, j}
\end{gather*}
The above mathematical formula (or description of computation) is a
closed form solution for the sum over exponentially many
elements. Note that its complexity is linear in size of $W$, which is $O(n^2)$.

\subsubsection{$\bf g(x \rightarrow x^2, W)$}

Now we wish to compute the following expression: 
\begin{gather*}
	g(x \rightarrow x^2, W) = \sum_{v \in \{0, 1\}^n, h \in \{0, 1\}^m} (v^TWh)^2
\end{gather*}

There are multiple second order monomials that emerge: 

\begin{itemize}
	\item $w_{i,j}^2$ -- present iff $v_i = 1, h_j = 1$. Appears $2^{n + m - 2}$ times. We encode sum of all monomials like this as $(1, 0, 0, 0)$.
	\item $w_{i,j} w_{i, k}, j \neq k$ -- present iff $v_i = 1, h_j = 1, h_k = 1$. Appears $2^{n + m - 3}$ times. We encode sum of all monomials like this as $(0, 1, 0, 0)$.	
	\item $w_{i,j} w_{k, j}, i \neq k$ -- present iff $v_i = 1, v_k = 1, h_j = 1$. Appears $2^{n + m - 3}$ times. We encode sum of all monomials like this as $(0, 0, 1, 0)$.
	\item $w_{i,j} w_{k, l}, i \neq k, j \neq l$ -- present iff $v_i = 1, v_k = 1, h_j = 1, h_l = 1$. Appears $2^{n + m - 4}$ times. We encode sum of all monomials like this as $(0, 0, 0, 1)$.
\end{itemize}
We encode above quantities in a vector, which indicate how many times
each of the monomials 
appears. The vector expressing this relation for $g(x \mapsto x^2, W)$ is $(2^{n + m - 2}, 2^{n + m - 3}, 2^{n + m - 3}, 2^{n + m - 4})$

Rob: too big a jump. Need transition sentence here.
Let us consider the following expressions: 
\begin{itemize}
 \item $\sum_{i = 1, \dots, n, j = 1, \dots m} W_{i, j}^2$ encodes $(1, 0, 0, 0)$. 
This expression contains only type of second degree monomials. 
It contains only monomials $w_{i, j}^2$, but not $w_{i, j} w_{i, k}$, or $w_{i, j} w_{k, j}$, or $w_{i, j} w_{k, l}$.
 \item $(\sum_{i = 1, \dots, n, j = 1, \dots m} W_{i, j})^2$ encodes $(1, 1, 1, 1)$.
 \item $\sum_{i = 1, \dots, n}(\sum_{j = 1, \dots, m} W)^2$ encodes $(1, 1, 0, 0)$. 
 \item $\sum_{j = 1, \dots, m}(\sum_{i = 1, \dots, n} W)^2$ encodes $(1, 0, 1, 0)$.
\end{itemize}
 
 By solving the linear equations:
 \begin{equation}
 \begin{pmatrix} 
  1 & 1 & 1 & 1 \\ 
  0 & 1 & 1 & 0 \\ 
  0 & 1 & 0 & 1 \\ 
  0 & 1 & 0 & 0 \\     
\end{pmatrix}x = 2^{n + m - 4}\begin{pmatrix} 
  2^{2}\\ 
  2^{1}\\ 
  2^{1}\\ 
  2^{0}\\     
\end{pmatrix} \\
 \end{equation}

The solution to this equation is $x=[1, 1, 1, 1]^T$, thus 
\begin{align*}
	&g(x \rightarrow x^2, W) = 2^{n + m - 4} \\ 
 &\Big(\sum_{i = 1, \dots, n, j = 1, \dots m} W_{i, j}^2 + (\sum_{i = 1, \dots, n, j = 1, \dots m} W_{i, j})^2 + \\
 &\sum_{i = 1, \dots, n}(\sum_{j = 1, \dots, m} W_{i, j})^2 + \sum_{j = 1, \dots, m}(\sum_{i = 1, \dots, n} W_{i, j})^2 \Big)
\end{align*}
Our algorithm derived the following equivalent Matlab computation for it:
\begin{lstlisting}
(sum(sum(W)) .^ 2 + ...
sum(sum(W, 2) .* sum(W, 2)) + ... 
sum(sum(W, 1) .* sum(W, 1)) + ... 
sum(sum(W .* W))) * 2 ^ (n + m - 4)
\end{lstlisting}
The above derivation is still within the scope of human skill. However, manual derivation
of $g(x \rightarrow x^k, W)$ for $k > 2$ quickly becomes intractable. Our algorithm
is able to find such complex computational patterns automatically. 

\subsubsection{{$\bf g(x \rightarrow x^3, W)$}}
The manual derivation for $k = 3$ is challenging. We present all generated expressions up to degree 
$3$ (for notation clearity, we first define variables $A$-$E$).
\vspace{-0.4cm}
%We present here all derived computations in for expressions in $\mathcal{P}^{1 \times 1}_3$. 
\begin{lstlisting}
A := sum(W, 2);
B := sum(W, 1);
C := sum(sum(W));
D := repmat(sum(W, 1), [n, 1]);
E := repmat(sum(W, 2), [1, m]);

C, C .^ 2, C .^ 3, sum(B .^ 2), 
sum(B .^ 3), sum(A .^ 2), sum(A .^ 3)
sum(sum(B .*  E .* W))), sum(sum((W .* W)))
sum(sum(D .^ 2 .* E))
sum(B .* sum(W .* W, 1)))
C .* sum(sum((W .* W), 2))
sum(sum((E .^ 2 .* D)))) 
sum(A .* sum((W .* W), 2))
sum(sum(W .* W .* W))
\end{lstlisting}

%\begin{lstlisting}
%C
%C .^ 2
%sum(B .^ 2) 
%sum(sum(B .*  E .* W))) 
%sum(B .^ 3)
%sum(A .^ 2)
%sum(A .^ 3)
%sum(sum((W .* W)))
%C .^ 3
%sum(sum(D .^ 2 .* E))
%sum(B .* sum(W .* W, 1)))
%C .* sum(sum((W .* W), 2))
%sum(sum((E .^ 2 .* D)))) 
%sum(A .* sum((W .* W), 2))
%sum(sum(W .* W .* W))
%\end{lstlisting}
%
Forming a linear system of these expressions and solving, we obtain
the following expression for $g(x \rightarrow x^3, W)$, which is
exactly equivalent to the original:
\vspace{-0.3cm}
\begin{lstlisting}
(C .^ 3 + 
C .* sum(A .^ 2) * 3 +
C .* sum(sum(W .* W, 2)) * 3 +
C .* sum(B .^ 2) * 3 + 
sum(A .* sum(W .* D, 2)) * 6)) / 64;
\end{lstlisting}

% XXX : After such rewriting verify in code !!!!
% Change result as a vector by grammar multiplication.
% check if one is subset of the other.
