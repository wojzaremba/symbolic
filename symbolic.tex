\documentclass{article}

\usepackage{graphicx} % more modern
\usepackage{subfigure} 
\usepackage{amssymb}
\usepackage{amsmath}

% For citations
\usepackage{natbib}

% For algorithms
\usepackage{comment}
\usepackage{algorithm}
\usepackage{algorithmic}

\usepackage{hyperref}

\newcommand{\theHalgorithm}{\arabic{algorithm}}

\newtheorem{theorem}{Theorem}[section]
\newtheorem{lemma}[theorem]{Lemma}
\newtheorem{conjecture}[theorem]{Conjecture}
\newtheorem{proposition}[theorem]{Proposition}
\newtheorem{definition}[theorem]{Definition}
\newtheorem{corollary}[theorem]{Corollary}

\usepackage{icml2012} 


\begin{document} 

\icmltitle{Mathematical formulas discovery}

\icmlauthor{Wojciech Zaremba}{woj.zaremba@gmail.com}
\icmladdress{}

\icmlkeywords{machine learning, RBMs, NP, Boltzmann machine}

\vskip 0.3in


\begin{abstract} 
abstract 
\end{abstract} 


\section{Introduction} \label{introduction} 
Progress in some branches of
mathematics might be restricted by size of symbolic derivations that human can
handle.  Mathematicians can deal with expressions having up to some small
number of variables having different purposes in equations. However, there
might exists very handy relations between expressions having hundreds or
thousands of variables. We will show how task of finding equivalent
mathematical expressions can be automated. Our focus is on finding equivalent
mathematical formulas, which are faster to compute than the original one. Where
faster can be defined as (1) number of operations, or (2) computational time
for any particular computational hardware.


We develop deterministic framework, which discovers relations between
multi-variable polynomial expressions. First, we construct grammars of
admissible operations together with cost of every operation. Through linear
combination of grammar elements, we find solution to desired expressions that
we want to compute. Finally, we show that this computation solution can be
further speed up by use of standard techniques of optimization in compiler
domain. We show power of presented approach by deriving close form, linear time
solutions to compute partition function like expressions for RBMs. Presented
close form solutions oppose widespread believe that hardness of partition
function computation is in its exponential number of elements.


Finally, we look on set of generated rules as on small axiomatic system.
Presented algorithm is deterministic, finite time proof in this system.
We are excited about using machine learning for automatic reasoning, and
this subset of mathematics seem to be perfect fit for testing automatic reasoning systems.
Usually, axiomatic systems like number theory, set theory, or topology have very small
number of axioms, but their application to terms might be complex. Proofs
for such systems are difficult to represent in software, and there is no
baseline algorithms which could brute force proof by iterating over all of them in reasonable finite time.
Where in the contrary, presented here axiomatic system has very simple representation
in software, and this paper presents brute force algorithm to find proofs. 
Surprisingly, even brute force algorithm is able to find new more efficient,
computational expressions for expressions that we use.




\section{Related work} \label{relatedwork}

compilers, computer vision grammars, nlp grammars. Finding rules in physics.

\section{Admissible computation as a grammars}\label{sec:grammars}

We consider grammar on tuples : (expression, computation, computational time).
We focus attention on expressions, which are matrices of polynomials. We denote
matrix of size $n \times m$ of $k$-th degree polynomials by $\mathcal{P}^{n
\times m}_k$.  The final goal is to find tuple for a given expression having
a shortest computational time. Where computational time is any measure of
complexity, which we care about. For instance, for theoretical computer science
purposes it might be number of operations, or computation complexity. However,
from perspective of engineering, it might be computation time on specific GPU
model. 


Let's ground our approach in an example. We assume that we are interest in
finding algorithm with smallest operation complexity, and computation platform
is a Matlab.  We consider following set of operations on matrices : 

\begin{align*}
&\text{{\bf Element wise multiplication}}\\
&(a \in \mathcal{P}^{n \times m}_\alpha, A, t_1), (b \in \mathcal{P}^{n \times m}_\beta, B, t_2) \rightarrow \\ 
&(c \in \mathcal{P}^{n \times m}_{\alpha + \beta}, A .* B, O(t_1 + t_2 + nm)) \\
&c_{i,j} = a_{i,j}b_{i, j} \text{ for } i \in \{1, \dots, n\}, j \in \{1, \dots, m\} \\
&\text{{\bf Matrix multiplication}}\\
&(a \in \mathcal{P}^{n \times m}_\alpha, A, t_1), (b \in \mathcal{P}^{m \times k}_\beta, B, t_2) \rightarrow \\ 
&(c \in \mathcal{P}^{n \times k}_{\alpha + \beta}, A * B, O(t_1 + t_2 + nmk)) \\
&c_{i,k} = \sum_{k = 1}^m a_{i,k}b_{k, j} \text{ for } i \in \{1, \dots, n\}, j \in \{1, \dots, m\} \\
&\text{{\bf Columns marginalization}}\\
&(a \in \mathcal{P}^{n \times m}_\alpha, A, t_1) \rightarrow \\ 
&(c \in \mathcal{P}^{n \times 1}_\alpha, sum(A, 2), O(t_1 + nm)) \\
&c_{i, 1} = \sum_{j = 1}^m a_{i, j} \text{ for } i \in \{1, \dots, n\}\\
&\text{{\bf Rows marginalization}}\\
&(a \in \mathcal{P}^{n \times m}_\alpha, A, t_1) \rightarrow \\ 
&(c \in \mathcal{P}^{1 \times m}_\alpha, sum(A, 1), O(t_1 + nm)) \\
&c_{1, j} = \sum_{i = 1}^m a_{i, j} \text{ for } j \in \{1, \dots, m\}\\
&\text{{\bf Columns repetition}}\\
&(a \in \mathcal{P}^{n \times 1}_\alpha, A, t_1) \rightarrow \\ 
&(c \in \mathcal{P}^{n \times m}_\alpha, repmat(A, 1, m), O(t_1 + nm)) \\
&c_{i, j} = a_{i, 1} \text{ for } i \in \{1, \dots, n\}\\
&\text{{\bf Rows repetition}}\\
&(a \in \mathcal{P}^{1 \times m}_\alpha, A, t_1) \rightarrow \\ 
&(c \in \mathcal{P}^{n \times m}_\alpha, repmat(A, n, 1), O(t_1 + nm)) \\
&c_{i, j} = a_{1, j} \text{ for } j \in \{1, \dots, m\}\\
&\text{{\bf Entry repetition}}\\
&(a \in \mathcal{P}^{1 \times 1}_\alpha, A, t_1) \rightarrow \\ 
&(c \in \mathcal{P}^{n \times m}_\alpha, repmat(A, n, m), O(t_1 + nm)) \\
&c_{i, j} = a_{1, 1} \text{ for } i \in \{1, \dots, n\}, j \in \{1, \dots, m\}\\
&\text{{\bf Transposition}}\\
&(a \in \mathcal{P}^{n \times m}_\alpha, A, t_1) \rightarrow \\ 
&(c \in \mathcal{P}^{m \times n}_\alpha, A', O(t_1 + nm)) \\
&c_{j, i} = a_{i, j} \text{ for } i \in \{1, \dots, n\}, j \in \{1, \dots, m\} \\
\end{align*}

For theoretical analysis we always consider extended version of grammar which consist of two additional rules : 
\begin{align*}
&\text{{\bf Rule$^*$ - Addition}}\\
&(a \in \mathcal{P}^{n \times m}_\alpha, A, t_1), (b \in \mathcal{P}^{n \times m}_\beta, B, t_2) \rightarrow \\ 
&(c \in \mathcal{P}^{n \times m}_{max(\alpha, \beta)}, A + B, O(t_1 + t_2 + nm)) \\
&c_{i,j} = a_{i,j} + b_{i, j} \text{ for } i \in \{1, \dots, n\}, j \in \{1, \dots, m\} \\
&\text{{\bf Rule$^*$ - scalar multiplication}}\\
&(a \in \mathcal{P}^{n \times m}_\alpha, A, t_1), (\lambda \in \mathcal{P}^{1}_0, B, 0) \rightarrow \\ 
&(\lambda a \in \mathcal{P}^{n \times m}_{\alpha}, \lambda * A, O(t_1 + nm)) \\
\end{align*}

We are given expression $Y \in \mathcal{P}_k$, which we would like to compute
fast. Moreover, we are given $X \in \mathcal{P}^{n \times m}_1$, which we have
direct access to in zero time. We present algorithm, which can find computation
of $Y$ given $X$ assuming that grammar together with its extended set of rules
generate $Y$ from tuple $(X, X, 0)$.

In order to find cheap way of expressing $Y$ as function of $X$, we proceed as follows : 
\begin{itemize}
\item develop grammar to obtain all possible expressions up to degree $k$. It gives rise to finite number of expressions.
\item find linear combination of expressions which is equal to polynomial $Y$ (this relation is a symbolic relation, which is valid for any assignment of symbols $X$). This linear problem can be constrained to simultaneously find set of expressions which sum of computational times is smallest.
\item apply optimization like subexpression elimination, and exploit distributive property of multiplication in order to decrees final computational time of $Y$.
\end{itemize}

As long as $Y$ can be expressed using extended grammar, we have a guarantee that above procedure will find a computation of $Y$. However, solution might be suboptimal, as we are using
greedy algorithm. We believe that fining best solution is at least NP-hard problem. 

\subsection{Linear combination}

\subsection{Optimization}

Application of grammar rules gives rise to a tree. Moreover, last step of
combining elements together can be also consider as adding plus node to the
tree. Optimization procedures are applied recursively to tree branches, and
transform them to more efficient computationally equivalents. We used common in
compiler literature optimization techniques like subexpression elimination.
Further, we optimized our symbolic tree by exploiting (1) distributive, and (2)
associative properties of multiplication, and addition (together with
commutative property for addition).


\section{Partition function of RBM} Algorithm presented in Section
\ref{sec:grammars} allows to find concise formulas for polynomial expressions.
However, many interesting functions are outside of this family.  Instead of it,
we are going to consider Taylor expansion of desired function, and we will
derive close form fast formula for it.

Let's denote by $g(f, W)$ generalization of partition function as follows : 

\begin{gather*}
g(f, W) = \sum_{v \in \{0, 1\}^n, h \in \{0, 1\}^m} f(v^TWh) \\
f : \mathcal{R} \rightarrow \mathcal{R}\\
W \in \mathcal{R}^{n \times m}
\end{gather*}

We consider computation of $g(x \rightarrow x^k, W)$ for a given $k$ for any $W
\in \mathcal{R}^{n \times m}$, for any $n, m$. Potentially, if we would be able
to compute $g(x \rightarrow x^k, W)$ for $k = 1, \dots, K$, than partition
function for finite energy $v^TWh < C$ could be approximated arbitrarily well.
This is consequence of expressing $e^{x}$ as finite sum approximation through
Taylor expansion.

\subsection{Low degree examples} In order to present how our algorithm works,
we will manually derive fast computation procedure for $g(x \rightarrow k, W)$.
However, this can be done manually only for very small $k = 1, 2$. 


\subsubsection{$g(x \rightarrow x, W)$} Let's consider function $f(x) = x$. We
will show that function $g(x \mapsto x, W)$ is computable in $O(nm)$ time
(linear with respect to number of entries in $W$ matrix).
 
\begin{gather*}
	g(x \mapsto x, W) = \sum_{v \in \{0, 1\}^n, h \in \{0, 1\}^m} v^TWh
\end{gather*}

Entry $w_{i,j}$ in above sum is counted only if $v_i = 1$ and $h_j = 1$. Other variables
$v_1, \dots v_{i-1}, v_{i+1}, \dots v_n$ and $h_1, \dots h_{i-1}, h_{j+1}, \dots h_m$ can be 
assigned arbitrary. Number of arbitrary assignments of remaining variables is $2^{n + m - 2}$. 
This concludes that 

\begin{gather*}
	\sum_{v \in \{0, 1\}^n, h \in \{0, 1\}^m} v^TWh = 2^{n + m - 2}\sum_{i = 1, \dots, n, j = 1, \dots, m} W_{i, j}
\end{gather*}
, which is the close form solution for sum over exponentially many elements.

\subsubsection{$g(x \rightarrow x^2, W)$}

We are interest in computing following expression : 
\begin{gather*}
	g(x \mapsto x^2, W) = \sum_{v \in \{0, 1\}^n, h \in \{0, 1\}^m} (v^TWh)^2
\end{gather*}

There are multiple second order monomials that emerge: 

\begin{itemize}
	\item $w_{i,j}^2$ - present iff $v_i = 1, h_j = 1$. Appears $2^{n + m - 2}$ times.
	\item $w_{i,j} w_{i, k}, j \neq k$ - present iff $v_i = 1, h_j = 1, h_k = 1$. Appears $2^{n + m - 3}$ times.	
	\item $w_{i,j} w_{k, j}, i \neq k$ - present iff $v_i = 1, v_k = 1, h_j = 1$. Appears $2^{n + m - 3}$ times.
	\item $w_{i,j} w_{k, l}, i \neq k, j \neq l$ - present iff $v_i = 1, v_k = 1, h_j = 1, h_l = 1$. Appears $2^{n + m - 4}$ times.			
\end{itemize}
We encode above quantities in a vector, which indicate how many times particular monomials 
appear. Vector expressing this relation for $g(x \mapsto x^2, W)$ is $(2^{n + m - 2}, 2^{n + m - 3}, 2^{n + m - 3}, 2^{n + m - 4})$


Let us consider following expressions : 
\begin{itemize}
 \item $\sum_{i = 1, \dots, n, j = 1, \dots m} W_{i, j}^2$ encodes $(1, 0, 0, 0)$. 
 \item $(\sum_{i = 1, \dots, n, j = 1, \dots m} W_{i, j})^2$ encodes $(1, 1, 1, 1)$.
 \item $\sum_{i = 1, \dots, n}(\sum_{j = 1, \dots, m} W)^2$ encodes $(1, 1, 0, 0)$. 
 \item $\sum_{j = 1, \dots, m}(\sum_{i = 1, \dots, n} W)^2$ encodes $(1, 1, 0, 0)$. 
\end{itemize}
 
 By solving linear equation :
 \begin{equation}
 \begin{pmatrix} 
  1 & 1 & 1 & 1 \\ 
  0 & 1 & 1 & 0 \\ 
  0 & 1 & 0 & 1 \\ 
  0 & 1 & 0 & 0 \\     
\end{pmatrix}t = 2^{n + m}\begin{pmatrix} 
  2^{-2}\\ 
  2^{-3}\\ 
  2^{-3}\\ 
  2^{-4}\\     
\end{pmatrix} \\
 \end{equation}

One can find that 
\begin{align*}
	&g(x \mapsto x^2, W) = 2^{n + m} * \\ 
 &\Big( 2.5\sum_{i = 1, \dots, n, j = 1, \dots m} W_{i, j}^2 + 0.5(\sum_{i = 1, \dots, n, j = 1, \dots m} W_{i, j})^2 + \\
 &0.5\sum_{i = 1, \dots, n}(\sum_{j = 1, \dots, m} W)^2 + 0.5\sum_{j = 1, \dots, m}(\sum_{i = 1, \dots, n} W)^2 \Big)\\
\end{align*}
\begin{align}
(((\sum_{i = 1, \dots, m} \sum_{j = 1, \dots, m} W_{i, j}  .* \sum_{i = 1, \dots, m} \sum_{j = 1, \dots, m} W_{i, j})    W_{i, j}) \\
  + (\sum_{i = 1, \dots, m} ( \sum_{j = 1, \dots, m} W_{i, j} .* \sum_{j = 1, \dots, m} W_{i, j})    W_{i, j}) \\
  + ( \sum_{i = 1, \dots, m} \sum_{j = 1, \dots, m} ( W_{i, j} .* W_{i, j})   W_{i, j})  + (\sum_{i = 1, \dots, m} \sum_{j = 1, \dots, m} ( W_{i, j} .* repmat(\sum_{i = 1, \dots, m} W_{i, j}, [n, 1]))    W_{i, j}))\\
\end{align}
This derivation is conducted in $O(nm)$ time.


Above derivation is still in scope of human skills. However, manual derivation
of $g(x \mapsto x^k, W)$ for further $k > 2$  seems to be a feet. Our algorithm
is able to find such complex computational patterns automatically.

XXX : check above equation.

\section{Experiments}

\subsection{Summations over matrix multiplications}

$\sum AB$

\subsection{Partition function approximation}

\section{Research agenda}

\subsection{Computation generalization}

\subsection{Derivations in unbounded grammar domain}

\subsubsection{Guarantees}

\section{Summary}
\nocite{*}
\bibliography{bibliography}
\bibliographystyle{icml2012}

\end{document} 

